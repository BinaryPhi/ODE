\documentclass[12pt]{scrartcl}
\title{}
\nonstopmode
\usepackage{graphicx}			% Required for including pictures
\usepackage[figurename=Figure]{caption}
\usepackage{float}    				% For tables and other floats
\usepackage{amsmath}  			% For math
\usepackage{bbm}  				% For mathBBM
\usepackage{amssymb}  			% For more math
\usepackage{fullpage} 			% Set margins and place page numbers at bottom center
\usepackage{paralist} 			% paragraph spacing
\usepackage{listings} 			% For source code
\usepackage{enumitem} 			% useful for itemization
\usepackage{siunitx}  			% standardization of si units
\usepackage{tikz,bm} 			% Useful for drawing plots
\usepackage{fancyhdr}
\usepackage{setspace}
\usepackage{mathtools}
\usepackage{hyperref}
\usepackage{bm}

\newcommand*\circled[1]{\tikz[baseline=(char.base)]{
            \node[shape=circle,draw,inner sep=1.2pt] (char) {#1};}}

\usepackage{amsthm}
\usepackage[framemethod=TikZ]{mdframed}

\parindent=23pt

%%%%%%%%%%%%%%%%%%%%%%%%%%%%%%
%Form
\newcounter{form}[section]\setcounter{form}{0}
\newenvironment{form}[2][]{%
\refstepcounter{form}%
\ifstrempty{#1}%
{\mdfsetup{%
frametitle={%
\tikz[baseline=(current bounding box.east),outer sep=0pt]
\node[anchor=east,rectangle,fill=violet!40]
{\strut \textit{Form}};}}
}{}%
\mdfsetup{innertopmargin=5pt,linecolor=violet!25,%
linewidth=1.5pt,topline=true,%
frametitleaboveskip=\dimexpr-\ht\strutbox\relax
}
\begin{mdframed}[]\relax%
\label{#2}}{\end{mdframed}}
%%%%%%%%%%%%%%%%%%%%%%%%%%%%%%


%%%%%%%%%%%%%%%%%%%%%%%%%%%%%%
%Solution
\newcounter{sol}[section]\setcounter{prf}{0}
\newenvironment{sol}[2][]{%
\refstepcounter{sol}%
\ifstrempty{#1}%
{\mdfsetup{%
frametitle={%
\tikz[baseline=(current bounding box.east),outer sep=0pt]
\node[anchor=east,rectangle,fill=red!20]
{\strut \textit{Solution}};}}
}{}%
\mdfsetup{innertopmargin=5pt,linecolor=red!25,%
linewidth=1.5pt,topline=true,%
frametitleaboveskip=\dimexpr-\ht\strutbox\relax
}
\begin{mdframed}[]\relax%
\label{#2}}{\end{mdframed}}
%%%%%%%%%%%%%%%%%%%%%%%%%%%%%%

%%%%%%%%%%%%%%%%%%%%%%%%%%%%%%
%Example
\newcounter{egs}[section]\setcounter{egs}{0}
\newenvironment{egs}[2][]{%
\refstepcounter{egs}%
\ifstrempty{#1}%
{\mdfsetup{%
frametitle={%
\tikz[baseline=(current bounding box.east),outer sep=0pt]
\node[anchor=east,rectangle,fill=purple!30]
{\strut \textit{Example}};}}
}{}%
\mdfsetup{innertopmargin=5pt,linecolor=purple!25,%
linewidth=1.5pt,topline=true,%
frametitleaboveskip=\dimexpr-\ht\strutbox\relax
}
\begin{mdframed}[]\relax%
\label{#2}}{\end{mdframed}}
%%%%%%%%%%%%%%%%%%%%%%%%%%%%%%



\hypersetup{
    colorlinks=true,
    linkcolor=violet,
    urlcolor=violet,
}


\setstretch{1.2}

\renewcommand{\headrulewidth}{0pt}
\renewcommand\footrule{\hrule height1pt}

\pagestyle{fancy}
\fancyhf{}
\lfoot{Ordinary Differential Equation \raisebox{0.5\depth}{\scalebox{0.8}\textcopyright}~BinaryPhi}
\rfoot{Page \thepage}

\begin{document}

	\begin{center}
		\textbf{ \large Ordinary Differential Equation \raisebox{0.5\depth}{\scalebox{0.8}\textcopyright}~BinaryPhi}
	\end{center}
	\vspace{-0.3em}
	\noindent\textbf{Author:} Kami Mou, Jinwei Zou
	\hspace{\fill}  \textbf{Last Edit:} \today~PDT \vspace{-0.6em} \\
	\hrule

\vspace{3em}

%%%%%%%%%%%%%%%%TOC
\hyperlink{Sec. 1}{\fontfamily{lmss}\selectfont
	\hspace{-20pt} \textbf{\textcolor{red}{Sec. 1:} Separable Differential Equations (Solution and 2 Examples)}
}\\

\hyperlink{Sec. 2}{\fontfamily{lmss}\selectfont
	\hspace{-20pt} \textbf{\textcolor{red}{Sec. 2:} Homogeneous Equations (Solution and 1 Example)}
}\\


\newpage
%%%%%%%%%%%%%%%%1
%%%%%%%%%%%%%%%%Separable Differential Equations
%%%%%%%%%%%%%%%%1
\hypertarget{Sec. 1}{\fontfamily{lmss}\selectfont
	\noindent \textbf{\textcolor{red}{Sec. 1:} Separable Differential Equations} \label{Sec. 1}
}
\vspace{1em}	

\begin{form}{}
\vspace{-1em}			%DoNotDelete/ChangeWhenNecessary
\addtolength{\jot}{0.4em}	%DoNotDelete
%
\begin{equation}
\frac{dy}{dx}=f(x)g(y)\,.
\end{equation}
%
\vspace{-1em}				%DoNotDelete/ChangeWhenNecessary
\end{form}

\begin{sol}{}
\vspace{-1.5em}			%DoNotDelete/ChangeWhenNecessary
\addtolength{\jot}{0.4em}		%DoNotDelete
\paragraph{\textcolor{blue}{\textbf{1)}}}
%%%%%
%%%%%%%%%%%Indentation
\parshape
2
0pt \linewidth
1\parindent \dimexpr \linewidth-1\parindent
%%%%%%%%%%%Indentation
%%%%%
When $g(y) &\neq 0$,
%
\begin{align*}
\frac{dy}{g(y)}&=f(x)dx\,,\\
\int \frac{dy}{g(y)}&= \int f(x)dx + C\,.
\end{align*}
%

This is the \textcolor{blue}{General Solution} to this type of differential equations.

\paragraph{\textcolor{blue}{\textbf{2)}}}
\parshape
2
0pt \linewidth
\parindent \dimexpr \linewidth-\parindent
When $g(y) &= 0$, of all functions of $y$, if exists $y_0$, as a constant, such that 

$g(y)=g(y_0)=0,$ where $y_0$ has the property that 

$$\frac{dy}{dx}=0.$$

\vspace{0.5em}We can say that $y=y_0$ is also one of the solutions to the differential equation.
\vspace{0.5em}
\end{sol}

\newpage
\begin{egs}{1}
\vspace{-1em}			%DoNotDelete/ChangeWhenNecessary
\addtolength{\jot}{0.4em}	%DoNotDelete
%
\begin{equation*}
\text{tan}(x)\,\frac{dy}{dx}=1+y\,.
\end{equation*}
%
\vspace{-2em}			%DoNotDelete/ChangeWhenNecessary
\paragraph{\textcolor{blue}{\textbf{[Solution]}}}
\parshape
2
0pt \linewidth
2.85\parindent \dimexpr \linewidth-2.85\parindent
First, make this equation neat:
$$\frac{dy}{dx}=\frac{1}{\text{tan}(x)}\,(1+y)\,,$$
which is the form of the separable differential equation.\\
Then assume the function of $y$ to be not equal to $0$:
%
\begin{align*} % asterisk *
\frac{1}{1+y}\,dy&=\frac{1}{\text{tan}(x)}\,dx\,,\\ % ampersand &
\frac{1}{1+y}\,dy&=\frac{\text{cos}(x)}{\text{sin}(x)}\,dx\,,\\
\frac{1}{1+y}\,dy&=\frac{1}{\text{sin}(x)}\,d\left(\text{sin}(x)\right),\\
\int \frac{1}{1+y}\,dy&=\int \frac{1}{\text{sin}(x)}\,d\left(\text{sin}(x)\right)+\textcolor{red}{c}\,,\\
\text{ln}\left|1+y\right| &= \text{ln}\left|\text{sin}(x)\right|+\textcolor{red}{c}\,\vphantom{\frac{1}{1}}\\
&=\text{ln}\left|e^{\textcolor{red}{c}}\cdot\text{sin}(x)\right|,\vphantom{\frac{1}{1}}\\
\left|1+y\right| &= \left|e^{\textcolor{red}{c}}\cdot\text{sin}(x)\right|, \vphantom{\frac{1}{1}}\\
1+y &= \pm\, e^{\textcolor{red}{c}}\cdot\text{sin}(x),C := \pm\, e^{\textcolor{red}{c}} \vphantom{\frac{1}{1}}\\
y &= C\cdot\text{sin}(x)-1\,, \vphantom{\frac{1}{1}}
\end{align*}
%
where $C$ is not equal to $0$. Now, consider function $1+y$ to be equal to $0$:
$$1+y=0 \Longrightarrow y=-1\,.$$
$y=-1$ is, in fact, a solution to this equation that tan$(x)\,0=0.$ In addition, the assumption of $y=-1$ matches the circumstance when $C=0$.\\
Therefore, the \textcolor{blue}{general solution} of this differential equation is \vspace{-0.5em}
$$\Aboxed{y = C\cdot\text{sin}(x)-1, \forall C \in \mathbb{R}\,.}$$
\vspace{-21pt}
\end{egs}

\newpage
\begin{egs}{2}
\vspace{-1em}			%DoNotDelete/ChangeWhenNecessary
\addtolength{\jot}{0.4em}	%DoNotDelete
%
\begin{equation*}
dx+xydy=y^2dx+ydy\,.
\end{equation*}
%
\vspace{-2em}			%DoNotDelete/ChangeWhenNecessary
\paragraph{\textcolor{blue}{\textbf{[Solution]}}}
\parshape
2
0pt \linewidth
2.85\parindent \dimexpr \linewidth-2.85\parindent
First, make this equation neat:

\end{egs}

%\Aboxed{&AnswersHere}\\

\newpage
%%%%%%%%%%%%%%%%2
%%%%%%%%%%%%%%%%Homogeneous Equations
%%%%%%%%%%%%%%%%2


\end{document}














